% !TEX encoding = UTF-8 Unicode
\documentclass[a4paper]{easychair}

\usepackage[T1]{fontenc} %
\usepackage[english]{babel}
\usepackage[utf8x]{inputenc}

\usepackage{amsmath, amssymb, amsfonts, stmaryrd}
\usepackage{pifont}
\PrerenderUnicode{é} % For the author names in the heading

% Add some colors
\usepackage[usenames,dvipsnames,svgnames,table]{xcolor}
\usepackage{hyperref}
\hypersetup{
 linktocpage,
 colorlinks,
 citecolor=BlueViolet,
 filecolor=red,
 linkcolor=Blue,
 urlcolor=BrickRed
}

\usepackage{latex/agda}

\usepackage{graphicx}
\usepackage{placeins}

% Meta comment
\newcommand\meta[1]{\textcolor{blue}{\emph{#1}}}

% Include the macro file
\input{macros}

\title{Normalization by Evaluation for Sized Dependent Types}
\titlerunning{NbE for Sized Types}

\author{
  Andreas Abel\inst{1}
  \and
  Andrea Vezzosi\inst{1}
  \and
  Th\'{e}o Winterhalter\inst{2}
}
\authorrunning{Abel and Winterhalter}

\institute{
  Department of Computer Science and Eng., Gothenburg
  University, Sweden \\
  \texttt{\{abela,vezzosi\}@chalmers.se}
  \and
  \' Ecole Normale Supérieure de Cachan, France \\
  \texttt{theo.winterhalter@ens-cachan.fr}
}

\begin{document}

\maketitle

\begin{abstract}
Sized types have been developed to make termination checking more perspicuous, more powerful, and more modular by integrating termination into type checking.  In dependently-typed proof assistants where proofs by induction are just recursive functional programs, the termination checker is an integral component of the trusted core, as validity of proofs depend on termination.  However, a rigorous integration of full-fledged sized types into dependent type theory is lacking so far.  Such an integration is non-trivial, as explicit sizes in proof terms might get in the way of equality checking, making terms appear distinct that should have the same semantics.
%
In this work, we integrate dependent types and sized types with
higher-rank size polymorphism, which is essential for generic
programming and abstraction.  We introduce a size quantifier $\forall$
which lets us ignore sizes in terms for equality checking, alongside
with a second quantifier $\Pi$ for abstracting over sizes that do
affect the semantics of types and terms.  Judgmental equality is
decided by an adaptation of normalization-by-evaluation. % for our new type theory, which features \emph{type shape}-directed reflection and reification.
\end{abstract}


\medskip
%\section{The Need for an Irrelevant Size Quantifier}

Agda \cite{agdawiki} features first class size polymorphism
\cite{abel:fics12} in contrast to theoretical accounts of sized
dependent types
\cite{bartheGregoirePastawski:lpar06,blanqui:rta04,sacchini:lics13}
who typically just have prenex (ML-style) size quantification.
Consequently, Agda's internal language contains size expressions in
terms wherever a size quantifier is instantiated.  However, these
size expressions, which are not unique due to subtyping, can get in
the way of reasoning about sizeful programs.
Consider the type of sized natural numbers.

\AgdaHide{
\begin{code}%
\>[0]\AgdaSymbol{\{{-}\#}\AgdaSpace{}%
\AgdaKeyword{OPTIONS}\AgdaSpace{}%
\AgdaOption{{-}{-}experimental{-}irrelevance}\AgdaSpace{}%
\AgdaSymbol{\#{-}\}}\<%
\\
\>[0]\AgdaSymbol{\{{-}\#}\AgdaSpace{}%
\AgdaKeyword{OPTIONS}\AgdaSpace{}%
\AgdaOption{{-}{-}allow{-}unsolved{-}metas}\AgdaSpace{}%
\AgdaSymbol{\#{-}\}}\<%
\\
\>[0]\AgdaSymbol{\{{-}\#}\AgdaSpace{}%
\AgdaKeyword{OPTIONS}\AgdaSpace{}%
\AgdaOption{{-}{-}show{-}irrelevant}\AgdaSpace{}%
\AgdaSymbol{\#{-}\}}\<%
\\
%
\\
\>[0]\AgdaKeyword{open}\AgdaSpace{}%
\AgdaKeyword{import}\AgdaSpace{}%
\AgdaModule{Agda.Builtin.Size}\<%
\\
\>[0][@{}l@{\AgdaIndent{0}}]%
\>[2]\AgdaKeyword{public}\AgdaSpace{}%
\AgdaKeyword{using}\AgdaSpace{}%
\AgdaSymbol{(}\AgdaPostulate{Size}\AgdaSymbol{)}\AgdaSpace{}%
\AgdaKeyword{renaming}\AgdaSpace{}%
\AgdaSymbol{(}\AgdaPostulate{ω}\AgdaSpace{}%
\AgdaSymbol{to}\AgdaSpace{}%
\AgdaPostulate{∞}\AgdaSymbol{;}\AgdaSpace{}%
\AgdaPostulate{↑\_}\AgdaSpace{}%
\AgdaSymbol{to}\AgdaSpace{}%
\AgdaPostulate{\_+1}\AgdaSymbol{)}\<%
\\
%
\\
\>[0]\AgdaKeyword{open}\AgdaSpace{}%
\AgdaKeyword{import}\AgdaSpace{}%
\AgdaModule{Agda.Builtin.Nat}\<%
\\
\>[0][@{}l@{\AgdaIndent{0}}]%
\>[2]\AgdaKeyword{public}\AgdaSpace{}%
\AgdaKeyword{using}\AgdaSpace{}%
\AgdaSymbol{(}\AgdaInductiveConstructor{suc}\AgdaSymbol{)}\AgdaSpace{}%
\AgdaKeyword{renaming}\AgdaSpace{}%
\AgdaSymbol{(}\AgdaDatatype{Nat}\AgdaSpace{}%
\AgdaSymbol{to}\AgdaSpace{}%
\AgdaDatatype{ℕ}\AgdaSymbol{)}\<%
\\
%
\\
\>[0]\AgdaKeyword{open}\AgdaSpace{}%
\AgdaKeyword{import}\AgdaSpace{}%
\AgdaModule{Agda.Builtin.Equality}\<%
\\
%
\\
\>[0]\AgdaFunction{\_+\_}\AgdaSpace{}%
\AgdaSymbol{:}\AgdaSpace{}%
\AgdaPostulate{Size}\AgdaSpace{}%
\AgdaSymbol{→}\AgdaSpace{}%
\AgdaDatatype{ℕ}\AgdaSpace{}%
\AgdaSymbol{→}\AgdaSpace{}%
\AgdaPostulate{Size}\<%
\\
\>[0]\AgdaBound{s}\AgdaSpace{}%
\AgdaFunction{+}\AgdaSpace{}%
\AgdaNumber{0}\AgdaSpace{}%
\AgdaSymbol{=}\AgdaSpace{}%
\AgdaBound{s}\<%
\\
\>[0]\AgdaBound{s}\AgdaSpace{}%
\AgdaFunction{+}\AgdaSpace{}%
\AgdaInductiveConstructor{suc}\AgdaSpace{}%
\AgdaBound{n}\AgdaSpace{}%
\AgdaSymbol{=}\AgdaSpace{}%
\AgdaSymbol{(}\AgdaBound{s}\AgdaSpace{}%
\AgdaFunction{+}\AgdaSpace{}%
\AgdaBound{n}\AgdaSymbol{)}\AgdaSpace{}%
\AgdaPostulate{+1}\<%
\end{code}
}

\newcommand{\apred}{\AgdaFunction{pred}}
\newcommand{\amonus}{\AgdaFunction{monus}}
\newcommand{\arefl}{\AgdaInductiveConstructor{refl}}
\newcommand{\azero}{\AgdaInductiveConstructor{zero}}
\newcommand{\asuc}{\AgdaInductiveConstructor{suc}}
\newcommand{\aNat}{\AgdaDatatype{Nat}}

\begin{code}%
\>[0]\AgdaKeyword{data}\AgdaSpace{}%
\AgdaDatatype{Nat}\AgdaSpace{}%
\AgdaSymbol{:}\AgdaSpace{}%
\AgdaPostulate{Size}\AgdaSpace{}%
\AgdaSymbol{→}\AgdaSpace{}%
\AgdaPrimitiveType{Set}\AgdaSpace{}%
\AgdaKeyword{where}\<%
\\
\>[0][@{}l@{\AgdaIndent{0}}]%
\>[2]\AgdaInductiveConstructor{zero}%
\>[8]\AgdaSymbol{:}%
\>[11]\AgdaSymbol{∀}\AgdaSpace{}%
\AgdaBound{i}\AgdaSpace{}%
\AgdaSymbol{→}\AgdaSpace{}%
\AgdaDatatype{Nat}\AgdaSpace{}%
\AgdaSymbol{(}\AgdaBound{i}\AgdaSpace{}%
\AgdaFunction{+}\AgdaSpace{}%
\AgdaNumber{1}\AgdaSymbol{)}\<%
\\
\>[0][@{}l@{\AgdaIndent{0}}]%
\>[2]\AgdaInductiveConstructor{suc}%
\>[8]\AgdaSymbol{:}%
\>[11]\AgdaSymbol{∀}\AgdaSpace{}%
\AgdaBound{i}\AgdaSpace{}%
\AgdaSymbol{→}\AgdaSpace{}%
\AgdaDatatype{Nat}\AgdaSpace{}%
\AgdaBound{i}\AgdaSpace{}%
\AgdaSymbol{→}\AgdaSpace{}%
\AgdaDatatype{Nat}\AgdaSpace{}%
\AgdaSymbol{(}\AgdaBound{i}\AgdaSpace{}%
\AgdaFunction{+}\AgdaSpace{}%
\AgdaNumber{1}\AgdaSymbol{)}\<%
\end{code}

We define subtraction $x \dotminus y$ on natural numbers, sometimes
called the $\amonus$ function, which computes $\max(0, x-y)$.  It is
defined by induction on the size $j$ of the second argument $y$, while
the output is bounded by size $i$ of the first argument $x$.  (The
input-output relation of $\amonus$ is needed for a natural
implementation of Euclidean divison.)

\begin{code}%
\>[0]\AgdaFunction{monus}\AgdaSpace{}%
\AgdaSymbol{:}\AgdaSpace{}%
\AgdaSymbol{∀}\AgdaSpace{}%
\AgdaBound{i}\AgdaSpace{}%
\AgdaSymbol{→}\AgdaSpace{}%
\AgdaDatatype{Nat}\AgdaSpace{}%
\AgdaBound{i}\AgdaSpace{}%
\AgdaSymbol{→}\AgdaSpace{}%
\AgdaSymbol{∀}\AgdaSpace{}%
\AgdaBound{j}\AgdaSpace{}%
\AgdaSymbol{→}\AgdaSpace{}%
\AgdaDatatype{Nat}\AgdaSpace{}%
\AgdaBound{j}\AgdaSpace{}%
\AgdaSymbol{→}\AgdaSpace{}%
\AgdaDatatype{Nat}\AgdaSpace{}%
\AgdaBound{i}\<%
\\
\>[0]\AgdaFunction{monus}\AgdaSpace{}%
\AgdaBound{i}%
\>[16]\AgdaBound{x}%
\>[27]\AgdaSymbol{.(}\AgdaBound{j}\AgdaSpace{}%
\AgdaFunction{+}\AgdaSpace{}%
\AgdaNumber{1}\AgdaSymbol{)}\AgdaSpace{}%
\AgdaSymbol{(}\AgdaInductiveConstructor{zero}\AgdaSpace{}%
\AgdaBound{j}\AgdaSymbol{)}%
\>[47]\AgdaSymbol{=}%
\>[50]\AgdaBound{x}\<%
\\
\>[0]\AgdaFunction{monus}\AgdaSpace{}%
\AgdaSymbol{.(}\AgdaBound{i}\AgdaSpace{}%
\AgdaFunction{+}\AgdaSpace{}%
\AgdaNumber{1}\AgdaSymbol{)}%
\>[16]\AgdaSymbol{(}\AgdaInductiveConstructor{zero}\AgdaSpace{}%
\AgdaBound{i}\AgdaSymbol{)}%
\>[27]\AgdaSymbol{.(}\AgdaBound{j}\AgdaSpace{}%
\AgdaFunction{+}\AgdaSpace{}%
\AgdaNumber{1}\AgdaSymbol{)}\AgdaSpace{}%
\AgdaSymbol{(}\AgdaInductiveConstructor{suc}\AgdaSpace{}%
\AgdaBound{j}\AgdaSpace{}%
\AgdaBound{y}\AgdaSymbol{)}%
\>[47]\AgdaSymbol{=}%
\>[50]\AgdaInductiveConstructor{zero}\AgdaSpace{}%
\AgdaBound{i}\<%
\\
\>[0]\AgdaFunction{monus}\AgdaSpace{}%
\AgdaSymbol{.(}\AgdaBound{i}\AgdaSpace{}%
\AgdaFunction{+}\AgdaSpace{}%
\AgdaNumber{1}\AgdaSymbol{)}%
\>[16]\AgdaSymbol{(}\AgdaInductiveConstructor{suc}\AgdaSpace{}%
\AgdaBound{i}\AgdaSpace{}%
\AgdaBound{x}\AgdaSymbol{)}%
\>[27]\AgdaSymbol{.(}\AgdaBound{j}\AgdaSpace{}%
\AgdaFunction{+}\AgdaSpace{}%
\AgdaNumber{1}\AgdaSymbol{)}\AgdaSpace{}%
\AgdaSymbol{(}\AgdaInductiveConstructor{suc}\AgdaSpace{}%
\AgdaBound{j}\AgdaSpace{}%
\AgdaBound{y}\AgdaSymbol{)}%
\>[47]\AgdaSymbol{=}%
\>[50]\AgdaFunction{monus}\AgdaSpace{}%
\AgdaBound{i}\AgdaSpace{}%
\AgdaBound{x}\AgdaSpace{}%
\AgdaBound{j}\AgdaSpace{}%
\AgdaBound{y}\<%
\end{code}

We wish to prove that subtracting $x$ from itself yields $0$, by
induction on $x$.  The case $x = 0$ should be trivial, as $x \dotminus
0 = x$ by definition, hence, $0 \dotminus 0 = 0$.  A simple proof by
reflexivity should suffice.  However, the goal shows a mismatch
between size $\infty$ and size $i$ coming from the computation of
$\amonus\,(i + 1)\,(\azero\,i)\,(i + 1)\,(\azero\,i)$.


\begin{code}%
\>[0]\AgdaFunction{monus{-}diag}\AgdaSpace{}%
\AgdaSymbol{:}\AgdaSpace{}%
\AgdaSymbol{∀}\AgdaSpace{}%
\AgdaBound{i}\AgdaSpace{}%
\AgdaSymbol{→}\AgdaSpace{}%
\AgdaSymbol{(}\AgdaBound{x}\AgdaSpace{}%
\AgdaSymbol{:}\AgdaSpace{}%
\AgdaDatatype{Nat}\AgdaSpace{}%
\AgdaBound{i}\AgdaSymbol{)}\AgdaSpace{}%
\AgdaSymbol{→}\AgdaSpace{}%
\AgdaInductiveConstructor{zero}\AgdaSpace{}%
\AgdaPostulate{∞}\AgdaSpace{}%
\AgdaDatatype{≡}\AgdaSpace{}%
\AgdaFunction{monus}\AgdaSpace{}%
\AgdaBound{i}\AgdaSpace{}%
\AgdaBound{x}\AgdaSpace{}%
\AgdaBound{i}\AgdaSpace{}%
\AgdaBound{x}\<%
\\
\>[0]\AgdaFunction{monus{-}diag}\AgdaSpace{}%
\AgdaSymbol{.(}\AgdaBound{i}\AgdaSpace{}%
\AgdaFunction{+}\AgdaSpace{}%
\AgdaNumber{1}\AgdaSymbol{)}\AgdaSpace{}%
\AgdaSymbol{(}\AgdaInductiveConstructor{zero}\AgdaSpace{}%
\AgdaBound{i}\AgdaSymbol{)}%
\>[31]\AgdaSymbol{=}%
\>[34]\AgdaSymbol{\{! zero ∞ ≡ zero i !\}}%
\>[57]\AgdaComment{{-}{-} goal}\<%
\\
\>[0]\AgdaFunction{monus{-}diag}\AgdaSpace{}%
\AgdaSymbol{.(}\AgdaBound{i}\AgdaSpace{}%
\AgdaFunction{+}\AgdaSpace{}%
\AgdaNumber{1}\AgdaSymbol{)}\AgdaSpace{}%
\AgdaSymbol{(}\AgdaInductiveConstructor{suc}\AgdaSpace{}%
\AgdaBound{i}\AgdaSpace{}%
\AgdaBound{x}\AgdaSymbol{)}%
\>[31]\AgdaSymbol{=}%
\>[34]\AgdaFunction{monus{-}diag}\AgdaSpace{}%
\AgdaBound{i}\AgdaSpace{}%
\AgdaBound{x}\<%
\end{code}

The proof could be completed by an application of reflexivity if Agda
ignored sizes where they act as \emph{type argument}, \ie, in
constructors and term-level function applications, but not in types
where they act as regular argument, \eg, in $\aNat\,i$.


%\section{Relevant and Irrelevant Size Quantification}

The problem is solved by distinguishing relevant ($\Pi$) from irrelevant
($\forall$) size quantification.  The relevant quantifier is the usual
dependent function space over sizes.  In particular, the congruence
rule for size application requires matching size arguments:
\[
  \ru{\Gamma \der t = t' : \Pi i.\, T \qquad
      \Gamma \der a : \Size
    }{\Gamma \der t\,a = t'\,a : T[a/i]}
\]
Typically, the relevant quantifier is used in types of types, for
instance, in its non-dependent form, in $\Nat : \aSize \to \Set$.
In contrast, the irrelevant size quantifier is used in types of
programs and ignores sizes in size applications.  The rules for
application, while Church-style, \textit{de facto} implement Curry-style
quantification:
\[
  \ru{\Gamma \der t = t' : \forall i.\, T \qquad
      \resurrect\Gamma \der a,a',b : \Size
    }{\Gamma \der t\,a = t'\,a' : T[b/i]}
\qquad
  \ru{\Gamma \der t : \forall i.\, T \qquad
      \resurrect\Gamma \der a,b : \Size
    }{\Gamma \der t\,a : T[b/i]}
\]
Further, the size arguments are scoped in the \emph{resurrected}
\cite{pfenning:lics01} context
$\resurrect \Gamma$ and, thus, are allowed to mention irrelevant size
variables.  Those are introduced by irrelevant size abstraction and
marked by the $\div$-symbol in the context.
In contrast, the quantified size variable may occur relevantly in the
\emph{type}.
\[
  \ru{\Gamma, i \erof \Size \der t : T
    }{\Gamma \der \lambda i.\, t : \forall i.\, T}
\qquad
  \ru{\Gamma, i \of \Size \der T : \Set
    }{\Gamma \der \forall i.\, T : \Set}
\]

The lack of type unicity arising from the size application rule has
prevented us from adopting the usual incremental algorithm for
equality checking
\cite{harperPfenning:equivalenceLF,abelScherer:types10}.
However, we have succeeded to employ normalization by evaluation
%\cite{bergerSchwichtenberg:lics91,abelCoquandDybjer:lics07}
\cite{abelCoquandDybjer:lics07}
for deciding judgmental equality \cite{abelVezzosiWinterhalter:icfp17}.

\bibliographystyle{plain}
\bibliography{auto-types17} % ../article/biblio,

\end{document}
