\nonstopmode
%% For double-blind review submission
%%\documentclass[acmsmall%acmlarge%,review,anonymous
%%]{acmart}\settopmatter{printfolios=true}
%% For single-blind review submission
%\documentclass[acmlarge,review]{acmart}\settopmatter{printfolios=true}
%% For final camera-ready submission
\documentclass[acmsmall,screen]{acmart}\settopmatter{}

% For having two versions of a paper in one file.
% Stuff that does not fit into the short version can be encosed in \LONGVERSION{...}
\ifdefined\LONGVERSION
  \relax
\else
% short version:
\newcommand{\LONGVERSION}[1]{}
\newcommand{\SHORTVERSION}[1]{#1}
% % long version:
% \newcommand{\LONGVERSION}[1]{#1}
% \newcommand{\SHORTVERSION}[1]{}
% \newcommand{\SHORTVERSION}[1]{BEGIN~SHORT\ #1 \ END~SHORT}
\fi
\newcommand{\LONGSHORT}[2]{\LONGVERSION{#1}\SHORTVERSION{#2}}
\newcommand{\SHORTLONG}[2]{\SHORTVERSION{#1}\LONGVERSION{#2}}
\newcommand{\EXTENDED}[1]{}

%\usepackage[right]{showlabels}\renewcommand{\showlabelfont}{\small\ttfamily\color{gray}}

% controlling flushleft/center for math displays
% http://www.golatex.de/wechsel-zwischen-leqno-und-reqno-fleqn-uvm-t2488.html
\makeatletter
%\def\leqn{\tagsleft@true}
%\def\reqn{\tagsleft@false}
\def\fleq{\@fleqntrue\let\mathindent\@mathmargin \@mathmargin=\leftmargini}
\def\cneq{\@fleqnfalse}
\g@addto@macro{\endsubequations}{\addtocounter{equation}{-1}}
\makeatother

\usepackage[utf8]{inputenc}
\usepackage{mathtools} %mathrlap
\usepackage[all]{xy}

\usepackage[cal=boondoxo]{mathalfa}
\usepackage{calc} % \widthof

%\usepackage{latex/agda}

% NO EFFECT:
% \renewcommand{\AgdaOperator}    [1]
%     {\AgdaNoSpaceMath{\AgdaFontStyle{\textcolor{AgdaOperator}{\mathbf{#1}}}}}

%% Note: Authors migrating a paper from PACMPL format to traditional
%% SIGPLAN proceedings format should change 'acmlarge' to
%% 'sigplan,10pt'.


%% Some recommended packages.
\usepackage{booktabs}   %% For formal tables:
                        %% http://ctan.org/pkg/booktabs
\usepackage{subcaption} %% For complex figures with subfigures/subcaptions
                        %% http://ctan.org/pkg/subcaption

\setcopyright{rightsretained}
\acmJournal{PACMPL}
\acmYear{2019}
\acmVolume{5}
\acmNumber{POPL}
\acmArticle{1}
\acmMonth{1}
\acmDOI{10.1145/???}
\acmPrice{}

%% \makeatletter\if@ACM@journal\makeatother
%% %% Journal information (used by PACMPL format)
%% %% Supplied to authors by publisher for camera-ready submission
%% \acmJournal{PACMPL}
%% \acmVolume{1}
%% \acmNumber{1}
%% \acmArticle{33}
%% \acmYear{2017}
%% \acmMonth{9}
%% \acmDOI{10.1145/nnnnnnn.nnnnnnn}
%% \startPage{1}
%% \else\makeatother
%% %% Conference information (used by SIGPLAN proceedings format)
%% %% Supplied to authors by publisher for camera-ready submission
%% \acmConference[PL'17]{ACM SIGPLAN Conference on Programming Languages}{January 01--03, 2017}{New York, NY, USA}
%% \acmYear{2017}
%% \acmISBN{978-x-xxxx-xxxx-x/YY/MM}
%% \acmDOI{10.1145/nnnnnnn.nnnnnnn}
%% \startPage{1}
%% \fi

\makeatletter
\newenvironment{proof*}[1][\proofname]{\par
  \normalfont \topsep6\p@\@plus6\p@\relax
  \trivlist
  \item[\@proofindent\hskip\labelsep
        {\@proofnamefont #1\@addpunct{.}}]\ignorespaces
}{%
  \endtrivlist\@endpefalse
}
\makeatother

%% Copyright information
%% Supplied to authors (based on authors' rights management selection;
%% see authors.acm.org) by publisher for camera-ready submission
%%\setcopyright{none}             %% For review submission
%\setcopyright{acmcopyright}
%\setcopyright{acmlicensed}
%\setcopyright{rightsretained}
%\copyrightyear{2017}           %% If different from \acmYear


% \acmBadgeR[http://icfp17.sigplan.org/track/icfp-2017-Artifacts]{artifact_evaluated-functional.png}
% \acmBadgeR[http://icfp17.sigplan.org/track/icfp-2017-Artifacts]{artifact_evaluated-reusable.png}
% \acmBadgeL[http://icfp17.sigplan.org/track/icfp-2017-Artifacts]{artifact_available.png}
% \acmBadgeL[https://hackage.haskell.org/package/Sit]{artifact_available.png}

%% Bibliography style
\bibliographystyle{ACM-Reference-Format}
%% Citation style
%% Note: author/year citations are required for papers published as an
%% issue of PACMPL.
\citestyle{acmauthoryear}   %% For author/year citations

\input{macros}

\renewcommand{\forallT}[2]{\forall #1.\,#2}
\renewcommand{\sext}[3]{(#1,#3/#2)}
\renewcommand{\cext}[3]{#1,#2{:}#3}
\renewcommand{\AA}{\mathcal{A}}

\begin{document}

%% Title information
\title{A Predicative Semantics of Sized Dependent Types}
%\titlenote{with title note}             %% \titlenote is optional;
                                        %% can be repeated if necessary;
                                        %% contents suppressed with 'anonymous'
%\subtitle{Subtitle}                     %% \subtitle is optional
%\subtitlenote{with subtitle note}       %% \subtitlenote is optional;
                                        %% can be repeated if necessary;
                                        %% contents suppressed with 'anonymous'


%% Author information
%% Contents and number of authors suppressed with 'anonymous'.
%% Each author should be introduced by \author, followed by
%% \authornote (optional), \orcid (optional), \affiliation, and
%% \email.
%% An author may have multiple affiliations and/or emails; repeat the
%% appropriate command.
%% Many elements are not rendered, but should be provided for metadata
%% extraction tools.

%% Author with single affiliation.
\author{Andreas Abel}
%\authornote{with author1 note}          %% \authornote is optional;
                                        %% can be repeated if necessary
\orcid{0000-0003-0420-4492}             %% \orcid is optional
\affiliation{
  \department{Department of Computer Science and Engineering}
  \institution{Gothenburg University, Sweden}
  \streetaddress{Rännvägen 6b}
  \city{Göteborg}
%  \state{State1}
  \postcode{41296}
  \country{Sweden}
}
\email{andreas.abel@gu.se}          %% \email is recommended

\author{Henning Basold}
\affiliation{
  \institution{Lyon}
  \country{France}
}
\email{henning@basold.eu}

% %% Author with two affiliations and emails.
% \author{Andreas Vezzosi}
% \affiliation{
%   \department{Department of Computer Science and Engineering}
%   \institution{Chalmers University of Technology, Sweden}
%   \streetaddress{Rännvägen 6b}
%   \city{Göteborg}
% %  \state{State1}
%   \postcode{41296}
%   \country{Sweden}
% }
% \email{vezzosi@chalmers.se}          %% \email is recommended

% \author{Theo Winterhalter}
% \affiliation{
%   \institution{École normale supérieure Paris-Saclay, France}
%   \country{France}
% }
% \email{theo.winterhalter@gmail.com}

%% Paper note
%% The \thanks command may be used to create a "paper note" ---
%% similar to a title note or an author note, but not explicitly
%% associated with a particular element.  It will appear immediately
%% above the permission/copyright statement.
%\thanks{with paper note}                %% \thanks is optional
                                        %% can be repeated if necesary
                                        %% contents suppressed with 'anonymous'


%% Abstract
%% Note: \begin{abstract}...\end{abstract} environment must come
%% before \maketitle command
\begin{abstract}
\end{abstract}


%% 2012 ACM Computing Classification System (CSS) concepts
%% Generate at 'http://dl.acm.org/ccs/ccs.cfm'.
 \begin{CCSXML}
<ccs2012>
<concept>
<concept_id>10003752.10003790.10011740</concept_id>
<concept_desc>Theory of computation~Type theory</concept_desc>
<concept_significance>500</concept_significance>
</concept>
<concept>
<concept_id>10003752.10010124.10010125.10010130</concept_id>
<concept_desc>Theory of computation~Type structures</concept_desc>
<concept_significance>500</concept_significance>
</concept>
<concept>
<concept_id>10003752.10010124.10010138.10010142</concept_id>
<concept_desc>Theory of computation~Program verification</concept_desc>
<concept_significance>300</concept_significance>
</concept>
<concept>
<concept_id>10003752.10010124.10010131.10010134</concept_id>
<concept_desc>Theory of computation~Operational semantics</concept_desc>
<concept_significance>100</concept_significance>
</concept>
</ccs2012>
\end{CCSXML}

\ccsdesc[500]{Theory of computation~Type theory}
\ccsdesc[500]{Theory of computation~Type structures}
\ccsdesc[300]{Theory of computation~Program verification}
\ccsdesc[100]{Theory of computation~Operational semantics}
%% End of generated code


%% Keywords
%% comma separated list
\keywords{dependent types, eta-equality, normalization-by-evaluation, proof irrelevance, sized types, subtyping, universes.}  %% \keywords is optional
%\keywords{Dependent Types, Normalization-by-Evaluation, Proof Irrelevance, Sized Types, Subtyping, Universes}  %% \keywords is optional


%% \maketitle
%% Note: \maketitle command must come after title commands, author
%% commands, abstract environment, Computing Classification System
%% environment and commands, and keywords command.
\maketitle


\section{Term model via a logical predicate}

Note: The semantics in this section does not support an irrelevant
size quantifier.  Sizes are relevant in types and terms.

\subsection{Preliminaries}

Weak head evaluation $t \evalsto w$ is defined as usual.

Let $\Phi,\Psi$ range over size contexts.  We write $\Phi \der a$ if
$a$ is a valid size expression in size context $\Phi$.
% and $\phi,\psi$ over their valuations.
We write $\phi \in \Phi$ if $\phi$ is a finite map of the
size variables declared in the size context $\Phi$ to ordinals.
We write $\Phi \leq \Psi$ if $\Phi$ is an extension (a thinning) of
$\Psi$ and $\phi \in \Phi \leq \psi \in \Psi$ if $\phi(i) = \psi(i)$
for all $i \in \Psi$.

Let $\NE$ be the set of all neutral terms.  Let $n,N$ range over neutrals.

A semantic type $\NAT^\alpha$ of natural numbers
strictly below $\alpha$ is defined inductively as by the following
rules:
\[
  \ru{n \in \NE
    }{n \in \NAT^\alpha}
\qquad
  \rux{
     }{\zero a \in \NAT^{\alpha}
     }{\alpha {>} 0}
\qquad
  \rux{t \in \NAThat^{\beta}
     }{\suc a t \in \NAT^{\alpha}
     }{\alpha {>} \beta}
\]
Here,
$\NAThat^\alpha = \{ t \mid t \evalsto w \mand w \in \NAT^\alpha \}$.

A semantic type $\ON^\alpha$ of tree ordinals strictly below $\alpha$
is defined inductively as follows:
\[
  \ru{n \in \NE
    }{n \in \ON^\alpha}
\qquad
  \rux{
     }{\zero a \in \ON^{\alpha}
     }{\alpha {>} 0}
\qquad
  \rux{t \in \ONhat^{\beta}
     }{\suc a t \in \ON^{\alpha}
     }{\alpha {>} \beta}
\qquad
  \rux{t\,u \in \ONhat^{\beta} \mforall u \in \NAThat^\omega
     }{\lim a t \in ON^{\alpha}
     }{\alpha{>}\beta}
\]
Here,
$\ONhat^\alpha = \{ t \mid t \evalsto w \mand w \in \ON^\alpha \}$.

\subsection{An inductive-recursive model of types}

By induction on $\ell \in \NN$, we define inductively sound type codes
$A \in \TY \Psi \psi \ell$ and simultaneously their extension
$\EL \Psi \psi \ell A \subseteq \Tm$ by in recursion on
$A \in \TY \Psi \psi \ell$.
In these definitions, we make use of the notations:
\[
\begin{array}{lcl}
  T \in \TYhat \Psi \psi \ell
    & \defiff & T \evalsto A \mand A \in \TY \Psi \psi \ell
\\
  \ELhat \Psi \psi \ell T
    & = & \EL \Psi \psi \ell A \mwhere T \evalsto A
\end{array}
\]
Case for neutrals:
\begin{align*}
  & \ru{N \in \NE}{N \in \TY \Psi \psi \ell}
  &
  \EL \Psi \psi \ell N = \NE
\end{align*}
Case for natural numbers:
\begin{align*}
  & \ru{\Psi \der a
      }{\Nat a \in \TY \Psi \psi \ell}
  & \EL \Psi \psi \ell {\Nat a}
    = \NAT^{\psi(a)}
\end{align*}
Case for tree ordinals:
\begin{align*}
  & \ru{\Psi \der a
      }{\On a \in \TY \Psi \psi \ell}
  & \EL \Psi \psi \ell {\On a}
    = \ON^{\psi(a)}
\end{align*}
Case for universes:
\begin{align*}
  & \ru{\ell' < \ell
      }{\Univ{\ell'} \in \TY \Psi \psi \ell}
  & \EL \Psi \psi \ell{\Univ{\ell'}}
    =
    \TY \Psi \psi {\ell'}
\end{align*}
Case for the function type:
\begin{align*}
  & \ru{A \in \TYhat \Psi \psi \ell \qquad
        \subst u x B \in \TYhat \Psi \psi \ell
        \mforall u \in \ELhat \Psi \psi \ell A
      }{\funT x A B \in \TY \Psi \psi \ell}
\\
  & t \in \EL \Psi \psi \ell {\funT x A B}
    \defas
    t\,u \in \ELhat \Psi \psi \ell {\subst u x B}
    \mforall u \in \ELhat \Psi \psi \ell A
\\
\end{align*}
Case for size polymorphism:
\begin{align*}
  & \ru{\forall \phi \in \Phi \leq \psi \in \Psi,\
        \Phi \der a.\ \
        \subst a i A \in \TYhat \Phi \phi \ell
      }{\forallT i A \in \TY \Psi \psi \ell}
\\
  & t \in \EL \Psi \psi \ell {\forallT i A}
    \defas
    t\,a \in \ELhat \Phi \phi \ell {\subst a i A}
    \mforall \phi \in \Phi \leq \psi \in \Psi
    \mand \Phi \der a
\end{align*}

\begin{lemma}[Monotonicity]
  If $\phi \in \Phi \leq \psi \in \Psi$ then
  $\TY \Psi \psi \ell \subseteq \TY \Phi \phi \ell$ and
  $\EL \Psi \psi \ell A = \EL \Phi \phi \ell A$ for all
  $A \in \TY \Psi \psi \ell$.
\end{lemma}


\subsection{Normalization}

In the following, we show that sound type codes and their elements are
weakly normalizing.  We make use of the following facts.
\begin{lemma}[Properties of weak normalization] \bla
  \begin{enumerate}
  \item $\WN$ is closed under type constructors and introductions.
  \item If\/ $t\,x \in \WN$ then $t \in \WN$ (note: $x$ is a variable).
  \item If\/ $t \evalsto w$ then $t \in \WN$ iff $w \in WN$.
  \end{enumerate}
\end{lemma}
\begin{theorem}
  If $A \in \TYhat \Psi \psi \ell$ then $A \in \WN$ and
  $\widehat\NE \subseteq \ELhat \Psi \psi \ell A \subseteq \WN$.
\end{theorem}
\noindent
\begin{proof*}
By induction on $A \in \TYhat \Psi \psi \ell$.
\begin{caselist}
\nextcase $\forallT i A \in \TY \Psi \psi \ell$.
In this case, we actually make use of the Kripke construction in the
semantics of the size quantifier.  However, we could probably get away
without the Kripke construction and use the fact that
$\subst \infty i A \in \WN$ implies $\forallT i A \in \WN$ and that
$t\,\infty \in \WN$ implies $t \in \WN$.

Let $\alpha$ be any ordinal number and observe $\Psi,i \der i$.  By
induction hypothesis with $\Phi = (\Psi,i)$ and
$\phi = \sext \psi i \alpha$ and $a = i$ we have $A \in \WN$, thus,
$\forallT i A \in \WN$.

Next, assume $n$ neutral, then
$n \in \ELhat \Psi \phi \ell {\forallT i A}$ by induction hypothesis,
since $n\,a$ is neutral as well for any size expression $a$.
Finally, assume $t \in \ELhat \Psi \psi \ell {\forallT i A}$.  By
induction hypothesis with $\Phi = (\Psi,i)$ and
$\phi = \sext \psi i 0$ and $a = i$ we have $t\,i \in \WN$, thus,
$t \in \WN$.
\qed
\end{caselist}
\end{proof*}


\subsection{Fundamental theorem and soundness of typing}

Given a size context $\Psi$ and a valuation $\psi \in \Psi$, the a
context soundness predicate $\Gamma \in \CXT \Psi \psi$ and the set of
its valuations $\VAL \Psi \psi \Gamma$ (given $\Gamma \in \CXT \Psi \psi$)
are defined by recursion on
$\Gamma$ as follows:
\begin{align*}
  & \ru{}{\cempty \in \CXT \Psi \psi}
  & \VAL \Psi \psi \cempty = \{ \cempty \}
\\
  & \ru{\Gamma \in \CXT \Psi \psi \qquad
      }{\cext \Gamma i \Size \in \CXT \Psi \psi}
  & \VAL \Psi \psi (\cext \Gamma i \Size) = \{ (\gamma,a) \mid
      \gamma \in \VAL \Psi \psi \Gamma \mand
      \Psi \der a \}
\end{align*}
\begin{align*}
  & \ru{\Gamma \in \CXT \Psi \psi \qquad
        A\gamma \in \TYhat \Psi \psi \ell
        \mforall \gamma \in \VAL \Psi \psi \Gamma
      }{\cext \Gamma x A \in \CXT \Psi \psi}
\\
  & \VAL \Psi \psi (\cext \Gamma x A) = \{ (\gamma,u) \mid
      \gamma \in \VAL \Psi \psi \Gamma \mand
      u \in \ELhat \Psi \psi \ell {A\gamma}
\end{align*}

% Given a size context $\Psi$ and a valuation $\psi \in \Psi$, the
% interpretation $\DenG \Gamma \Psi \psi$ of a typing context $\Gamma$ is
% a set of $\Gamma$-substitutions
% defined inductively by the following rules.
% \begin{gather*}
%   \ru{
%     }{\sempty \in \DenG \cempty \Psi \psi}
% \qquad
%   \ru{\gamma \in \DenG \Gamma \Psi \psi \qquad
%       \Psi \der a
%     }{\sext \gamma i a \in \DenG{\cext \Gamma i \Size} \Psi \psi}
% \qquad
%   \ru{\gamma \in \DenG \Gamma \Psi \psi \qquad
%       A \gamma \in \TYhat \Psi \psi \ell \qquad
%       u \in \ELhat \Psi \psi \ell {A \gamma}
%     }{\sext \gamma u x \in \DenG{\cext \Gamma x A} \Psi \psi}
% \end{gather*}

We define semantic judgements as follows:
\begin{align*}
  & \models \Gamma
    & \defas
    & \Gamma \in \CXT \Psi \psi \mforall \Psi \mand \psi \in \Psi
\\
  & \Gamma \models A
    & \defas
    & \models \Gamma \mand
      A \gamma \in \TYhat \Psi \psi \ell \mforsome \ell
      \mforall \Psi \mand \psi \in \Psi
      \mand \gamma \in \VAL \Psi \psi \Gamma
\\
  & \Gamma \models t : A
    & \defas
    & \Gamma \models A \mand
      t \gamma \in \ELhat \Psi \psi \ell {A \gamma}
      \mforall \Psi \mand \psi \in \Psi
      \mand \gamma \in \VAL \Psi \psi \Gamma
\\
  & \Gamma \models A \leq B
    & \defas
    & \Gamma \models A \mand \Gamma \models B \mand
      \ELhat \Psi \psi \ell A \subseteq \ELhat \Psi \psi \ell B
\end{align*}
\begin{lemma}
  If\/ $\models \cext \Gamma x A$ then $\Gamma \models A$.
\end{lemma}
\begin{lemma}[Subtyping laws]
  The following laws, written as inference rules, hold:
\begin{gather*}
  \ru{\Gamma \models A
    }{\Gamma \models A \leq A}
\qquad
  \ru{\Gamma \models A \leq B \qquad
      \Gamma \models B \leq C
    }{\Gamma \models A \leq C}
\qquad
  \ru{\Gamma \der a \leq b
    }{\Gamma \models \Nat a \leq \Nat b}
\qquad
  \ru{\Gamma \der a \leq b
    }{\Gamma \models \On a \leq \On b}
\\[2ex]
  \ru{\Gamma \models A' \leq A \qquad
      \cext \Gamma x {A'} \models B \leq B'
    }{\Gamma \models \funT x A B \leq \funT x {A'} {B'}}
\qquad
  \ru{\cext \Gamma i \Size \models A \leq A'
    }{\Gamma \models \forallT i A \leq \forallT i {A'}}
  \end{gather*}
\end{lemma}
\begin{theorem}[Typing laws]
  The following laws, written as inference rules, hold:
\begin{gather*}
  \ru{\Gamma \models t : A \qquad
      \Gamma \models A \leq B
    }{\Gamma \models t : B}
\\[2ex]
  \ru{\models \Gamma
    }{\Gamma \models x : \Gamma(x)}
\qquad
  \ru{\cext \Gamma x A \models t : B
    }{\Gamma \models \lambda x t : \funT x A B}
\qquad
  \ru{\Gamma \models t : \funT x A B \qquad
      \Gamma \models u : A
    }{\Gamma \models t\,u : \subst u x B}
\\[2ex]
  \ru{\cext \Gamma i \Size \models t : A
    }{\Gamma \models \lambda i t : \forallT i A}
\qquad
  \ru{\Gamma \models t : \forallT i A \qquad
      \Gamma \der a
    }{\Gamma \models t\,a : \subst a i A}
\qquad
  \dots
\end{gather*}
\end{theorem}
\noindent
\begin{proof*}
\begin{caselist}
\nextcase
\[
  \ru{\cext \Gamma i \Size \models t : A
    }{\Gamma \models \lambda i t : \forallT i A}
\]
Assume $\Psi$ and $\psi \in \Psi$ and
$\gamma \in \VAL \Psi \psi \Gamma$.  Further, assume $\Phi \leq \Psi$
and $\phi \in \Phi \leq \psi \in \Psi$ and $\Phi \der a$.  We show
$\subst a i A \in \TYhat \Phi \phi \ell$ and
$(\lambda i t)\,a \in \ELhat \Phi \phi \ell {\subst a i A}$.  By
monotonicity, $\gamma \in \VAL \Phi \phi \Gamma$, thus
$\sext \gamma i a \in \VAL \Phi \phi {(\cext \Gamma i \Size)}$.
By induction hypothesis $\subst a i A \in \TYhat \Phi \phi \ell$,
and $\subst a i t \in \ELhat \Phi \phi \ell {\subst a i A}$,
and we conclude by weak head expansion.

\nextcase
\[
  \ru{\Gamma \models t : \forallT i A \qquad
      \Gamma \der a
    }{\Gamma \models t\,a : \subst a i A}
\]
Assume $\Psi$ and $\psi \in \Psi$ and
$\gamma \in \VAL \Psi \psi \Gamma$.  We show
$(\subst a i A)\gamma \in \TYhat \Psi \psi \ell$ and
$(t\,a)\gamma \in \ELhat \Psi \psi \ell {(\subst a i A)\gamma}$.  By
induction hypothesis, $(\forall i A) \gamma \in \TY \Psi \psi \ell$
and $t\gamma \in \ELhat \Psi \psi \ell {(\forallT i A)\gamma}$.  Since
$\Psi \der a \gamma$, we have
$\subst {a\gamma} i {A\gamma} \in \TYhat \Psi \psi \ell$ and
$t \gamma\, (a \gamma) \in \ELhat \Psi \psi \ell {\subst {a\gamma} i
  {A\gamma}}$.
The rest is substitution laws.
\qed
\end{caselist}
\end{proof*}
Since the models does not have any notion of equality of terms, it
does not allow us to give a complete set of rules for type equality
and subtyping.
Thus, we refine the model.


\section{Term model via partial equivalence relations}

If $\AA$ is a partial equivalence relation (PER) on terms, We write $t = t'
\in \AA$ if $(t,t') \in \AA$, and $t \in \AA$ if $(t,t) \in \AA$.

Let $\NE$ be PER of neutrals such that $n = n' \in \NE$ iff $n$ and
$n'$ are identical except for size annotations $\ann a$.

The PER $\NAT^\alpha$ of natural numbers
strictly below $\alpha$ is defined inductively as by the following
rules:
\[
  \ru{n = n' \in \NE
    }{n = n' \in \NAT^\alpha}
\qquad
  \rux{
     }{\zero a = \zero {a'} \in \NAT^{\alpha}
     }{\alpha {>} 0}
\qquad
  \rux{t = t' \in \NAThat^{\beta}
     }{\suc a t = \suc{a'}{t'} \in \NAT^{\alpha}
     }{\alpha {>} \beta}
\]
Here,
$\NAThat^\alpha = \{ (t,t') \mid t \evalsto w \mand t' \evalsto
w' \mand w = w' \in \NAT^\alpha \}$.

A PER $\ON^\alpha$ of tree ordinals strictly below $\alpha$
is defined inductively as follows:
\begin{gather*}
  \ru{n = n' \in \NE
    }{n = n' \in \ON^\alpha}
\qquad
  \rux{
     }{\zero a = \zero{a'} \in \ON^{\alpha}
     }{\alpha {>} 0}
\qquad
  \rux{t = t' \in \ONhat^{\beta}
     }{\suc a t = \suc{a'}{t'} \in \ON^{\alpha}
     }{\alpha {>} \beta}
\\[2ex]
  \rux{t\,u = t'\,u' \in \ONhat^{\beta} \mforall u = u' \in \NAThat^\omega
     }{\lim a t = \lim{a'}{t'} \in ON^{\alpha}
     }{\alpha{>}\beta}
\end{gather*}
Here,
$\ONhat^\alpha  = \{ (t,t') \mid t \evalsto w \mand \mand t' \evalsto
w' and w = w' \in \ON^\alpha \}$.


By induction on $\ell \in \NN$, we define inductively a PER
$\TY \Psi \psi \ell$ of type codes and simultaneously their extension
$\EL \Psi \psi \ell A \subseteq \Tm$ by in recursion on
$A \in \TY \Psi \psi \ell$.
In these definitions, we make use of the notations:
\[
\begin{array}{lcl}
  T = T' \in \TYhat \Psi \psi \ell
    & \defiff & T \evalsto A \mand T' \evalsto A' \mand A = A' \in \TY \Psi \psi \ell
\\
  \ELhat \Psi \psi \ell T
    & = & \EL \Psi \psi \ell A \mwhere T \evalsto A
\end{array}
\]
Case for neutrals:
\begin{align*}
  & \ru{N = N' \in \NE}{N \in \TY \Psi \psi \ell}
  &
  \EL \Psi \psi \ell N = \NE
\end{align*}
Case for natural numbers:
\begin{align*}
  & \ru{\Psi \der a
      }{\Nat a = \Nat a \in \TY \Psi \psi \ell}
  & \EL \Psi \psi \ell {\Nat a}
    = \NAT^{\psi(a)}
\end{align*}
Case for tree ordinals:
\begin{align*}
  & \ru{\Psi \der a
      }{\On a = \On a \in \TY \Psi \psi \ell}
  & \EL \Psi \psi \ell {\On a}
    = \ON^{\psi(a)}
\end{align*}
Case for universes:
\begin{align*}
  & \ru{\ell' < \ell
      }{\Univ{\ell'} = \Univ{\ell'} \in \TY \Psi \psi \ell}
  & \EL \Psi \psi \ell{\Univ{\ell'}}
    =
    \TY \Psi \psi {\ell'}
\end{align*}
Case for the function type:
\begin{align*}
  & \ru{A = A' \in \TYhat \Psi \psi \ell \qquad
        \subst u x B = \subst{u'}x{B'} \in \TYhat \Psi \psi \ell
        \mforall u = u' \in \ELhat \Psi \psi \ell A
      }{\funT x A B = \funT x {A'} {B'} \in \TY \Psi \psi \ell}
\\
  & t = t' \in \EL \Psi \psi \ell {\funT x A B}
    \defas
    t\,u = t'\,u' \in \ELhat \Psi \psi \ell {\subst u x B}
    \mforall u = u' \in \ELhat \Psi \psi \ell A
\\
\end{align*}
Case for size polymorphism:
\begin{align*}
  & \ru{\forall \phi \in \Phi \leq \psi \in \Psi,\
        \Phi \der a.\ \
        \subst a i A = \subst a i {A'} \in \TY \Phi \phi \ell
      }{\forallT i A = \forallT i {A'} \in \TY \Psi \psi \ell}
\\
  & t = t' \in \EL \Psi \psi \ell {\forallT i A}
    \defas
    t\,a = t'\,a' \in \EL \Phi \phi \ell {\subst b i A}
    \mforall \phi \in \Phi \leq \psi \in \Psi
    \mand \Phi \der a,a',b
\end{align*}


\subsection{Fundamental theorem}




%% Bibliography
\bibliography{all}


% %% Appendix
% \appendix
% \section{Appendix}

% Text of appendix \ldots

\end{document}

%%% Local Variables:
%%% mode: latex
%%% TeX-master: "pred-sized"
%%% End:
