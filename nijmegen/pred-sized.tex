\nonstopmode
%% For double-blind review submission
%%\documentclass[acmsmall%acmlarge%,review,anonymous
%%]{acmart}\settopmatter{printfolios=true}
%% For single-blind review submission
%\documentclass[acmlarge,review]{acmart}\settopmatter{printfolios=true}
%% For final camera-ready submission
\documentclass[acmsmall,screen]{acmart}\settopmatter{}

% For having two versions of a paper in one file.
% Stuff that does not fit into the short version can be encosed in \LONGVERSION{...}
\ifdefined\LONGVERSION
  \relax
\else
% short version:
\newcommand{\LONGVERSION}[1]{}
\newcommand{\SHORTVERSION}[1]{#1}
% % long version:
% \newcommand{\LONGVERSION}[1]{#1}
% \newcommand{\SHORTVERSION}[1]{}
% \newcommand{\SHORTVERSION}[1]{BEGIN~SHORT\ #1 \ END~SHORT}
\fi
\newcommand{\LONGSHORT}[2]{\LONGVERSION{#1}\SHORTVERSION{#2}}
\newcommand{\SHORTLONG}[2]{\SHORTVERSION{#1}\LONGVERSION{#2}}
\newcommand{\EXTENDED}[1]{}

%\usepackage[right]{showlabels}\renewcommand{\showlabelfont}{\small\ttfamily\color{gray}}

% controlling flushleft/center for math displays
% http://www.golatex.de/wechsel-zwischen-leqno-und-reqno-fleqn-uvm-t2488.html
\makeatletter
%\def\leqn{\tagsleft@true}
%\def\reqn{\tagsleft@false}
\def\fleq{\@fleqntrue\let\mathindent\@mathmargin \@mathmargin=\leftmargini}
\def\cneq{\@fleqnfalse}
\g@addto@macro{\endsubequations}{\addtocounter{equation}{-1}}
\makeatother

\usepackage[utf8]{inputenc}
\usepackage{mathtools} %mathrlap
\usepackage[all]{xy}

\usepackage[cal=boondoxo]{mathalfa}
\usepackage{calc} % \widthof

%\usepackage{latex/agda}

% NO EFFECT:
% \renewcommand{\AgdaOperator}    [1]
%     {\AgdaNoSpaceMath{\AgdaFontStyle{\textcolor{AgdaOperator}{\mathbf{#1}}}}}

%% Note: Authors migrating a paper from PACMPL format to traditional
%% SIGPLAN proceedings format should change 'acmlarge' to
%% 'sigplan,10pt'.


%% Some recommended packages.
\usepackage{booktabs}   %% For formal tables:
                        %% http://ctan.org/pkg/booktabs
\usepackage{subcaption} %% For complex figures with subfigures/subcaptions
                        %% http://ctan.org/pkg/subcaption

\setcopyright{rightsretained}
\acmJournal{PACMPL}
\acmYear{2019}
\acmVolume{5}
\acmNumber{POPL}
\acmArticle{1}
\acmMonth{1}
\acmDOI{10.1145/???}
\acmPrice{}

%% \makeatletter\if@ACM@journal\makeatother
%% %% Journal information (used by PACMPL format)
%% %% Supplied to authors by publisher for camera-ready submission
%% \acmJournal{PACMPL}
%% \acmVolume{1}
%% \acmNumber{1}
%% \acmArticle{33}
%% \acmYear{2017}
%% \acmMonth{9}
%% \acmDOI{10.1145/nnnnnnn.nnnnnnn}
%% \startPage{1}
%% \else\makeatother
%% %% Conference information (used by SIGPLAN proceedings format)
%% %% Supplied to authors by publisher for camera-ready submission
%% \acmConference[PL'17]{ACM SIGPLAN Conference on Programming Languages}{January 01--03, 2017}{New York, NY, USA}
%% \acmYear{2017}
%% \acmISBN{978-x-xxxx-xxxx-x/YY/MM}
%% \acmDOI{10.1145/nnnnnnn.nnnnnnn}
%% \startPage{1}
%% \fi

\makeatletter
\newenvironment{proof*}[1][\proofname]{\par
  \normalfont \topsep6\p@\@plus6\p@\relax
  \trivlist
  \item[\@proofindent\hskip\labelsep
        {\@proofnamefont #1\@addpunct{.}}]\ignorespaces
}{%
  \endtrivlist\@endpefalse
}
\makeatother

%% Copyright information
%% Supplied to authors (based on authors' rights management selection;
%% see authors.acm.org) by publisher for camera-ready submission
%%\setcopyright{none}             %% For review submission
%\setcopyright{acmcopyright}
%\setcopyright{acmlicensed}
%\setcopyright{rightsretained}
%\copyrightyear{2017}           %% If different from \acmYear


% \acmBadgeR[http://icfp17.sigplan.org/track/icfp-2017-Artifacts]{artifact_evaluated-functional.png}
% \acmBadgeR[http://icfp17.sigplan.org/track/icfp-2017-Artifacts]{artifact_evaluated-reusable.png}
% \acmBadgeL[http://icfp17.sigplan.org/track/icfp-2017-Artifacts]{artifact_available.png}
% \acmBadgeL[https://hackage.haskell.org/package/Sit]{artifact_available.png}

%% Bibliography style
\bibliographystyle{ACM-Reference-Format}
%% Citation style
%% Note: author/year citations are required for papers published as an
%% issue of PACMPL.
\citestyle{acmauthoryear}   %% For author/year citations

\input{macros}

\renewcommand{\forallT}[2]{\forall #1.\,#2}



\begin{document}

%% Title information
\title{A Predicative Semantics of Sized Dependent Types}
%\titlenote{with title note}             %% \titlenote is optional;
                                        %% can be repeated if necessary;
                                        %% contents suppressed with 'anonymous'
%\subtitle{Subtitle}                     %% \subtitle is optional
%\subtitlenote{with subtitle note}       %% \subtitlenote is optional;
                                        %% can be repeated if necessary;
                                        %% contents suppressed with 'anonymous'


%% Author information
%% Contents and number of authors suppressed with 'anonymous'.
%% Each author should be introduced by \author, followed by
%% \authornote (optional), \orcid (optional), \affiliation, and
%% \email.
%% An author may have multiple affiliations and/or emails; repeat the
%% appropriate command.
%% Many elements are not rendered, but should be provided for metadata
%% extraction tools.

%% Author with single affiliation.
\author{Andreas Abel}
%\authornote{with author1 note}          %% \authornote is optional;
                                        %% can be repeated if necessary
\orcid{0000-0003-0420-4492}             %% \orcid is optional
\affiliation{
  \department{Department of Computer Science and Engineering}
  \institution{Gothenburg University, Sweden}
  \streetaddress{Rännvägen 6b}
  \city{Göteborg}
%  \state{State1}
  \postcode{41296}
  \country{Sweden}
}
\email{andreas.abel@gu.se}          %% \email is recommended

\author{Henning Basold}
\affiliation{
  \institution{Lyon}
  \country{France}
}
\email{henning@basold.eu}

% %% Author with two affiliations and emails.
% \author{Andreas Vezzosi}
% \affiliation{
%   \department{Department of Computer Science and Engineering}
%   \institution{Chalmers University of Technology, Sweden}
%   \streetaddress{Rännvägen 6b}
%   \city{Göteborg}
% %  \state{State1}
%   \postcode{41296}
%   \country{Sweden}
% }
% \email{vezzosi@chalmers.se}          %% \email is recommended

% \author{Theo Winterhalter}
% \affiliation{
%   \institution{École normale supérieure Paris-Saclay, France}
%   \country{France}
% }
% \email{theo.winterhalter@gmail.com}

%% Paper note
%% The \thanks command may be used to create a "paper note" ---
%% similar to a title note or an author note, but not explicitly
%% associated with a particular element.  It will appear immediately
%% above the permission/copyright statement.
%\thanks{with paper note}                %% \thanks is optional
                                        %% can be repeated if necesary
                                        %% contents suppressed with 'anonymous'


%% Abstract
%% Note: \begin{abstract}...\end{abstract} environment must come
%% before \maketitle command
\begin{abstract}
\end{abstract}


%% 2012 ACM Computing Classification System (CSS) concepts
%% Generate at 'http://dl.acm.org/ccs/ccs.cfm'.
 \begin{CCSXML}
<ccs2012>
<concept>
<concept_id>10003752.10003790.10011740</concept_id>
<concept_desc>Theory of computation~Type theory</concept_desc>
<concept_significance>500</concept_significance>
</concept>
<concept>
<concept_id>10003752.10010124.10010125.10010130</concept_id>
<concept_desc>Theory of computation~Type structures</concept_desc>
<concept_significance>500</concept_significance>
</concept>
<concept>
<concept_id>10003752.10010124.10010138.10010142</concept_id>
<concept_desc>Theory of computation~Program verification</concept_desc>
<concept_significance>300</concept_significance>
</concept>
<concept>
<concept_id>10003752.10010124.10010131.10010134</concept_id>
<concept_desc>Theory of computation~Operational semantics</concept_desc>
<concept_significance>100</concept_significance>
</concept>
</ccs2012>
\end{CCSXML}

\ccsdesc[500]{Theory of computation~Type theory}
\ccsdesc[500]{Theory of computation~Type structures}
\ccsdesc[300]{Theory of computation~Program verification}
\ccsdesc[100]{Theory of computation~Operational semantics}
%% End of generated code


%% Keywords
%% comma separated list
\keywords{dependent types, eta-equality, normalization-by-evaluation, proof irrelevance, sized types, subtyping, universes.}  %% \keywords is optional
%\keywords{Dependent Types, Normalization-by-Evaluation, Proof Irrelevance, Sized Types, Subtyping, Universes}  %% \keywords is optional


%% \maketitle
%% Note: \maketitle command must come after title commands, author
%% commands, abstract environment, Computing Classification System
%% environment and commands, and keywords command.
\maketitle


\section{Term model via a logical predicate}

Weak head evaluation $t \evalsto w$ is defined as usual.

Let $\Phi,\Psi$ range over size contexts.  We write $\Phi \der a$ if
$a$ is a valid size expression in size context $\Phi$.
% and $\phi,\psi$ over their valuations.
We write $\phi \in \Phi$ if $\phi$ is a finite map of the
size variables declared in the size context $\Phi$ to ordinals.
We write $\Phi \leq \Psi$ if $\Phi$ is an extension (a thinning) of
$\Psi$ and $\phi \in \Phi \leq \psi \in \Psi$ if $\phi(i) = \psi(i)$
for all $i \in \Psi$.

Let $\NE$ be the set of all neutral terms.  Let $n,N$ range over neutrals.

A semantic type $\NAT^\alpha$ of natural numbers
strictly below $\alpha$ is defined inductively as by the following
rules:
\[
  \ru{n \in \NE
    }{n \in \NAT^\alpha}
\qquad
  \rux{
     }{\zero a \in \NAT^{\alpha}
     }{\alpha {>} 0}
\qquad
  \rux{t \in \NAThat^{\beta}
     }{\suc a t \in \NAT^{\alpha}
     }{\alpha {>} \beta}
\]
Here,
$\NAThat^\alpha = \{ t \mid t \evalsto w \mand w \in \NAT^\alpha \}$.

A semantic type $\ON^\alpha$ of tree ordinals strictly below $\alpha$
is defined inductively as follows:
\[
  \ru{n \in \NE
    }{n \in \ON^\alpha}
\qquad
  \rux{
     }{\zero a \in \ON^{\alpha}
     }{\alpha {>} 0}
\qquad
  \rux{t \in \ONhat^{\beta}
     }{\suc a t \in \ON^{\alpha}
     }{\alpha {>} \beta}
\qquad
  \rux{t\,u \in \ONhat^{\beta} \mforall u \in \NAThat^\omega
     }{\lim a t \in ON^{\alpha}
     }{\alpha{>}\beta}
\]
Here,
$\ONhat^\alpha = \{ t \mid t \evalsto w \mand w \in \ON^\alpha \}$.

By induction on $\ell \in \NN$, we define inductively sound type codes
$A \in \TY \Psi \psi \ell$ and simultaneously their extension
$\EL \Psi \psi \ell A \subseteq \Tm$ by in recursion on
$A \in \TY \Psi \psi \ell$.
In these definitions, we make use of the notations:
\[
\begin{array}{lcl}
  T \in \TYhat \Psi \psi \ell
    & \defiff & T \evalsto A \mand A \in \TY \Psi \psi \ell
\\
  \ELhat \Psi \psi \ell T
    & = & \EL \Psi \psi \ell A \mwhere T \evalsto A
\end{array}
\]
Case for neutrals:
\begin{align*}
  & \ru{N \in \NE}{N \in \TY \Psi \psi \ell}
  &
  \EL \Psi \psi \ell N = \NE
\end{align*}
Case for natural numbers:
\begin{align*}
  & \ru{\Psi \der a
      }{\Nat a \in \TY \Psi \psi \ell}
  & \EL \Psi \psi \ell {\Nat a}
    = \NAT^{\psi(a)}
\end{align*}
Case for tree ordinals:
\begin{align*}
  & \ru{\Psi \der a
      }{\On a \in \TY \Psi \psi \ell}
  & \EL \Psi \psi \ell {\On a}
    = \ON^{\psi(a)}
\end{align*}
Case for universes:
\begin{align*}
  & \ru{\ell' < \ell
      }{\Univ{\ell'} \in \TY \Psi \psi \ell}
  & \EL \Psi \psi \ell{\Univ{\ell'}}
    =
    \TY \Psi \psi {\ell'}
\end{align*}
Case for the function type:
\begin{align*}
  & \ru{A \in \TYhat \Psi \psi \ell \qquad
        \subst u x B \in \TYhat \Psi \psi \ell
        \mforall u \in \ELhat \Psi \psi \ell A
      }{\funT x A B \in \TY \Psi \psi \ell}
\\
  & t \in \EL \Psi \psi \ell {\funT x A B}
    \defas
    t\,u \in \ELhat \Psi \psi \ell {\subst u x B}
    \mforall u \in \ELhat \Psi \psi \ell A
\\
\end{align*}
Case for size polymorphism:
\begin{align*}
  & \ru{\forall \phi \in \Phi \leq \psi \in \Psi,\
        \Phi \der a.\ \
        \subst a i A \in \TY \Phi \phi \ell
      }{\forallT i A \in \TY \Psi \psi \ell}
\\
  & t \in \EL \Psi \psi \ell {\forallT i A}
    \defas
    t\,a \in \EL \Phi \phi \ell {\subst a i A}
    \mforall \phi \in \Phi \leq \psi \in \Psi
    \mand \Phi \der a
\end{align*}



\section{Term model via partial equivalence relations}

If $\AA$ is a partial equivalence relation (PER) on terms, We write $t = t'
\in \AA$ if $(t,t') \in \AA$, and $t \in \AA$ if $(t,t) \in \AA$.

Let $\NE$ be PER of neutrals such that $n = n' \in \NE$ iff $n$ and
$n'$ are identical except for size annotations $\ann a$.

The PER $\NAT^\alpha$ of natural numbers
strictly below $\alpha$ is defined inductively as by the following
rules:
\[
  \ru{n = n' \in \NE
    }{n = n' \in \NAT^\alpha}
\qquad
  \rux{
     }{\zero a = \zero {a'} \in \NAT^{\alpha}
     }{\alpha {>} 0}
\qquad
  \rux{t = t' \in \NAThat^{\beta}
     }{\suc a t = \suc{a'}{t'} \in \NAT^{\alpha}
     }{\alpha {>} \beta}
\]
Here,
$\NAThat^\alpha = \{ (t,t') \mid t \evalsto w \mand \mand t' \evalsto
w' and w = w' \in \NAT^\alpha \}$.

A PER $\ON^\alpha$ of tree ordinals strictly below $\alpha$
is defined inductively as follows:
\begin{gather*}
  \ru{n = n' \in \NE
    }{n = n' \in \ON^\alpha}
\qquad
  \rux{
     }{\zero a = \zero{a'} \in \ON^{\alpha}
     }{\alpha {>} 0}
\qquad
  \rux{t = t' \in \ONhat^{\beta}
     }{\suc a t = \suc{a'}{t'} \in \ON^{\alpha}
     }{\alpha {>} \beta}
\\[2ex]
  \rux{t\,u = t'\,u' \in \ONhat^{\beta} \mforall u = u' \in \NAThat^\omega
     }{\lim a t = \lim{a'}{t'} \in ON^{\alpha}
     }{\alpha{>}\beta}
\end{gather*}
Here,
$\ONhat^\alpha  = \{ (t,t') \mid t \evalsto w \mand \mand t' \evalsto
w' and w = w' \in \ON^\alpha \}$.


By induction on $\ell \in \NN$, we define inductively a PER
$\TY \Psi \psi \ell$ of type codes and simultaneously their extension
$\EL \Psi \psi \ell A \subseteq \Tm$ by in recursion on
$A \in \TY \Psi \psi \ell$.
In these definitions, we make use of the notations:
\[
\begin{array}{lcl}
  T = T' \in \TYhat \Psi \psi \ell
    & \defiff & T \evalsto A \mand T' \evalsto A' \mand A = A' \in \TY \Psi \psi \ell
\\
  \ELhat \Psi \psi \ell T
    & = & \EL \Psi \psi \ell A \mwhere T \evalsto A
\end{array}
\]
Case for neutrals:
\begin{align*}
  & \ru{N = N' \in \NE}{N \in \TY \Psi \psi \ell}
  &
  \EL \Psi \psi \ell N = \NE
\end{align*}
Case for natural numbers:
\begin{align*}
  & \ru{\Psi \der a
      }{\Nat a = \Nat a \in \TY \Psi \psi \ell}
  & \EL \Psi \psi \ell {\Nat a}
    = \NAT^{\psi(a)}
\end{align*}
Case for tree ordinals:
\begin{align*}
  & \ru{\Psi \der a
      }{\On a = \On a \in \TY \Psi \psi \ell}
  & \EL \Psi \psi \ell {\On a}
    = \ON^{\psi(a)}
\end{align*}
Case for universes:
\begin{align*}
  & \ru{\ell' < \ell
      }{\Univ{\ell'} = \Univ{\ell'} \in \TY \Psi \psi \ell}
  & \EL \Psi \psi \ell{\Univ{\ell'}}
    =
    \TY \Psi \psi {\ell'}
\end{align*}
Case for the function type:
\begin{align*}
  & \ru{A = A' \in \TYhat \Psi \psi \ell \qquad
        \subst u x B = \subst{u'}x{B'} \in \TYhat \Psi \psi \ell
        \mforall u = u' \in \ELhat \Psi \psi \ell A
      }{\funT x A B = \funT x {A'} {B'} \in \TY \Psi \psi \ell}
\\
  & t = t' \in \EL \Psi \psi \ell {\funT x A B}
    \defas
    t\,u = t'\,u' \in \ELhat \Psi \psi \ell {\subst u x B}
    \mforall u = u' \in \ELhat \Psi \psi \ell A
\\
\end{align*}
Case for size polymorphism:
\begin{align*}
  & \ru{\forall \phi \in \Phi \leq \psi \in \Psi,\
        \Phi \der a.\ \
        \subst a i A = \subst a i {A'} \in \TY \Phi \phi \ell
      }{\forallT i A = \forallT i {A'} \in \TY \Psi \psi \ell}
\\
  & t = t' \in \EL \Psi \psi \ell {\forallT i A}
    \defas
    t\,a = t'\,a' \in \EL \Phi \phi \ell {\subst b i A}
    \mforall \phi \in \Phi \leq \psi \in \Psi
    \mand \Phi \der a,a',b
\end{align*}




%% Bibliography
\bibliography{all}


% %% Appendix
% \appendix
% \section{Appendix}

% Text of appendix \ldots

\end{document}

%%% Local Variables:
%%% mode: latex
%%% TeX-master: "pred-sized"
%%% End:
