\nonstopmode
\documentclass[t]{beamer}
%\documentclass{article}\usepackage{beamerarticle}

\makeatletter
%\def\leqn{\tagsleft@true}
%\def\reqn{\tagsleft@false}
\def\fleq{\@fleqntrue\let\mathindent\@mathmargin \@mathmargin=\leftmargini}
\def\cneq{\@fleqnfalse}
\g@addto@macro{\endsubequations}{\addtocounter{equation}{-1}}
\makeatother

\mode<presentation>
{
  % \usetheme{AnnArbor}
  \useinnertheme[shadow=true]{rounded}
  \useoutertheme{infolines}
  %\setbeamercolor*{title in head/foot}{parent=palette secondary}
  %\useoutertheme{shadow}
  \usecolortheme{wolverine}

  \setbeamerfont{block title}{size={}}
  \setbeamercolor{titlelike}{parent=structure,bg=yellow!85!orange}

  % My modification: centered title with white background
  \setbeamercolor{frametitle}{bg=white}
  \setbeamertemplate{frametitle}{
    \begin{center}
      \Large\insertframetitle
      \par
    \end{center}
  }
  % plus red emphasis
  \setbeamercolor{alerted text}{fg=red!80!black}

  % More space between items
  %\defbeamertemplate*{itemize/enumerate body begin}{default}{\itemsep1ex}


  \setbeamercolor{math text}{parent=titlelike}
%  \setbeamercolor{math text displayed}{parent=palette primary}

  % Fix background of theorems/proof
  \setbeamercolor{block body}{parent=palette primary}%{bg=yellow!85!orange}
  \setbeamercolor{block title}{parent=palette secondary}%{bg=orange}



  \setbeamercovered{transparent}
  % or whatever (possibly just delete it)

} % end mode presentation


\usepackage[english]{babel}
\usepackage[utf8x]{inputenc}
%\usepackage{times}  % DYSFUNCTIONAL! looks awful when mixing mathtt and ordinary math.
\usepackage{ifthen}
%\usepackage[T1]{fontenc} %no effect
% Or whatever. Note that the encoding and the font should match. If T1
% does not look nice, try deleting the line with the fontenc.
\usepackage{amsmath}
\usepackage{amssymb}
\usepackage{stmaryrd}
\usepackage{alltt}
\usepackage[normalem]{ulem} % strikethrough \sout
\usepackage{cancel} % strikethrough math mode \cancel
%\usepackage{enumitem} % set label in itemize
\usepackage{pifont} % for \tickNo
%\usepackage{}
%\usepackage{listings}
%\usepackage[all]{xy}
%\usepackage{proof}
%\usepackage{eurosym}
%\usepackage{graphics}
\usepackage{bibentry}
%\nobibliography{short}
\bibliographystyle{plain}
%\input{prooftree}

\usepackage{latex/agda}
\AgdaNoSpaceAroundCode{}

\DeclareMathSymbol{\chkmark}{\mathord}{AMSa}{"58}

% RGB colors
\definecolor{darkred}{rgb}{0.5,0,0}
\definecolor{darkgreen}{rgb}{0,0.5,0}
\definecolor{darkblue}{rgb}{0,0,0.5}
\definecolor{dirtyred}{rgb}{0.7,0.2,0.1}
\definecolor{dirtygreen}{rgb}{0.2,0.4,0.1}
\definecolor{darkdirtygreen}{rgb}{0.13,0.25,0.07}
\definecolor{dirtyblue}{rgb}{0.07,0.2,0.5}
\definecolor{darkdirtyblue}{rgb}{0.1,0.15,0.35}
\definecolor{lightblue}{rgb}{0.5,0.5,1}
\definecolor{olivegreen}{rgb}{0.5,0.5,0}
\definecolor{brown}{rgb}{0.65,0.35,0} % almost gold
\definecolor{grey}{rgb}{0.33,0.33,0.33}
\definecolor{darkbrown}{rgb}{0.35,0.15,0}
\definecolor{darkgrey}{rgb}{0.16,0.16,0.16}

\newcommand{\textred}[1]{{\color{dirtyred}\textbf{#1}}}

%\newcommand{\tickYes}{{\color{darkgreen}\checkmark}} % does not work
  %since checkmark invokes math mode
\newcommand{\tickYes}{\ensuremath{\color{darkgreen}\chkmark}}
\newcommand{\tickNo}{{\color{dirtyred}\hspace{1pt}\ding{55}}}

%\usepackage[curve,matrix,arrow]{xy}
%\usepackage{tikz} % Drawing diagrams
%\usepackage{pgflibraryshapes} %Ellipses


% black text in mbox
\newcommand{\mybox}[1]{\mbox{\color{black}#1}}

% MACROS:
\newcommand{\inst}{}
\input{macros}
% \renewcommand{\tin}{\mathsf{inn}}
% \renewcommand{\thead}{\cop{\mathsf{head}}}
% \renewcommand{\ttail}{\cop{\mathsf{tail}}}
% % start copattern color
% \newcommand{\shead}{\cop{\mathsf{head}}}
% \newcommand{\stail}{\cop{\mathsf{tail}}}
% \newcommand{\phead}{\cop{\mathsf{.head}}}
% \newcommand{\ptail}{\cop{\mathsf{.tail}}}
% \newcommand{\vecQ}{\cop{\vec Q}}
% \newcommand{\bDelta}{\boring{\Delta}}
% \newcommand{\bcovered}{\mathrel{\boring{\ccovered}}}
% \newcommand{\bA}{\boring{A}}
% \newcommand{\bC}{\boring{C}}
% \newcommand{\bder}{\mathrel{\boring{\vdash}}}
% \newcommand{\bcolon}{\mathrel{\boring{:}}}
% \newcommand{\blpar}{\boring{(}}
% \newcommand{\brpar}{\boring{)}}
% \renewcommand{\vbranches}{\boring{\mathit{branches}}}
% \renewcommand{\tout}{\cop{\mathsf{out}}}
% \renewcommand{\tforce}{\cop{\mathsf{force}}}
% \newcommand{\kw}[1]{{\bf\texttt{#1}}}
% \newcommand{\kwnewtype}{\kw{newtype}}
% \newcommand{\kwrecord}{\kw{record}}
% \newcommand{\kwwhere}{\kw{where}}
% \newcommand{\kwinstance}{\kw{instance}}
% \newcommand{\kwcoinductive}{\kw{coinductive}}
% \newcommand{\kwCoInductive}{\kw{CoInductive}}
% \newcommand{\kwCoFixpoint}{\kw{CoFixpoint}}
% \newcommand{\kwfield}{\kw{field}}
% \renewcommand{\tcase}{\mathbf{case}}
% \renewcommand{\tof}{\mathbf{of}}
% \renewcommand{\caseof}[3]{\tcase\,#1\,\tof~#2 \mathrel{\boldsymbol\To} #3}
% \newcommand{\tthead}{\cop{\texttt{head}}}
% \newcommand{\tttail}{\cop{\texttt{tail}}}
\newcommand{\ONE}{\mathbf{1}}

% DOES NOT WORK:
% rule with "boring" default text
\newcommand{\setboring}{\setbeamercolor{math text}{fg=grey}\setbeamercolor{math display}{fg=grey}}
\newcommand{\bru}[2]{\setboring\dfrac{\setboring#1}{\setboring#2}}

\renewcommand*\ttdefault{txtt} % for listing package

% \newcommand{\defHaskelllistings}{%
%   \lstset{%
%     language=Haskell,%
%     basicstyle=\ttfamily\small\color{darkdirtyblue},% \ttfamily
%     keywordstyle=\ttfamily\bfseries,% \underbar
%     identifierstyle=,%
%     commentstyle=\itshape,%
%     columns=flexible,%spaceflexible,% fixed,% flexible,%
%     showstringspaces=false,%
% %    xleftmargin=\codeindent,% defined below
%     breaklines=true,%
%     deletekeywords={succ,zero,head,tail,zipWith,Either,List},%
%     morekeywords={Set,Size,fun,cofun,pattern},% ,left,right,nil,cons
%     literate={\\}{{$\lambda$}}1 {->}{{$\rightarrow$~}}2
%              {<=}{{$\leq$~}}2 {<}{{$<$~}}1
% %     literate={map}{map~}4
% %       {even}{even~}5
% %       {odd}{odd~}4
%      }%
% }
%\defHaskelllistings

\title[NbE for Sized Dependent Types]{%
  Normalization by Evaluation for Sized Dependent Types}

\author[Abel Vezzosi Winterhalter]{
  Andreas Abel\inst{1}
  \and Andrea Vezzosi\inst{1}
  \and Theo Winterhalter\inst{2}
}
% \author{Andreas Abel\inst{1}
%   \and Brigitte Pientka\inst{2}
%   \and David Thibodeau\inst{2}
%   \and Anton Setzer\inst{3}
% }
%{F.~Author\inst{1} \and S.~Another\inst{2}}
% - Give the names in the same order as the appear in the paper.
% - Use the \inst{?} command only if the authors have different
%   affiliation.

\institute[] %Chalmers/GU/ENS Cachan] % (optional, but mostly needed)
{
  \inst{1}
  Department of Computer Science and Engineering\\
  Chalmers and Gothenburg University, Sweden \\[1ex]

  \inst{2}
  École Normale Supérieure de Cachan, France \\
}
%  \inst{1}%
%  Department of Computer Science\\
%  University of Somewhere
%  \and
%  \inst{2}%
%  Department of Theoretical Philosophy\\
%  University of Elsewhere}
%% - Use the \inst command only if there are several affiliations.
%% - Keep it simple, no one is interested in your street address.

\date[TYPES'17] % (optional, should be abbreviation of conference name)
{ 23rd International Conference on
  Types for Proofs and Programs \\
  TYPES 2017 \\
  Budapest, Hungary, 30 May 2017
}
% - Either use conference name or its abbreviation.
% - Not really informative to the audience, more for people (including
%   yourself) who are reading the slides online

%\subject{Software Verification}
% This is only inserted into the PDF information catalog. Can be left
% out.



% If you have a file called "university-logo-filename.xxx", where xxx
% is a graphic format that can be processed by latex or pdflatex,
% resp., then you can add a logo as follows:

% \pgfdeclareimage[height=0.5cm]{university-logo}{university-logo-filename}
% \logo{\pgfuseimage{university-logo}}



% Delete this, if you do not want the table of contents to pop up at
% the beginning of each subsection:

%\AtBeginSubsection[]
%\AtBeginSection[]
%{
%  \begin{frame}<beamer>
%    \frametitle{Outline}
%    \tableofcontents[currentsection,currentsubsection]
%  \end{frame}
%}


% If you wish to uncover everything in a step-wise fashion, uncomment
% the following command:

%\beamerdefaultoverlayspecification{<+->}

%\newcommand{\List}{\mathsf{List}}
\renewcommand{\Set}{\mathsf{Set}}
\newcommand{\Tree}{\mathsf{Tree}}
%\newcommand{\Prop}{\mathsf{Prop}}
%\newcommand{\Type}{\mathsf{Type}}
%\newcommand{\Size}{\mathsf{Size}}
%\newcommand{\tfix}{\mathsf{fix}}
%\newcommand{\Int}{\mathsf{Int}}
%\newcommand{\tnil}{\mathsf{nil}}
%\newcommand{\tcons}{\mathsf{cons}}
%\newcommand{\vas}{\mathit{as}}
%\newcommand{\ttuple}[1]{(#1)}

%\newcommand{\oford}{\of\tord} \newcommand{\oftype}{\of\ttype}
\newcommand{\oford}{}         \newcommand{\oftype}{}
\newenvironment{prg}{\begin{quotation}\begin{tabbing}}{\end{tabbing}\end{quotation}}

% COLORS
\newcommand{\cHead}{\color{darkblue}}
\newcommand{\cSub}{\color{brown}}
\newcommand{\cWhite}{\color{white}}
\newcommand{\cGray}{\color{gray}}
\newcommand{\cGreen}{\color{olivegreen}}
\newcommand{\cBrown}{\color{brown}}
\newcommand{\cBlack}{\color{black}}
\newcommand{\black}[1]{{\cBlack#1}}

\newcommand{\cAnn}{\color{red!80!black}}%purple darkblue
\newcommand{\cAside}{\color{gray}}
\newcommand{\cEnum}{\color{darkgreen}}
\newcommand{\cEm}{\cAnn} %\color{red}}
\newcommand{\cCo}{\cAnn} %\color{red}} % copattern color
\newcommand{\cop}[1]{{\cCo#1}}
\newcommand{\cApp}{\color{violet}}
\newcommand{\capp}[1]{{\cApp#1}}
\newcommand{\cFocus}{\color{darkgreen}}
\newcommand{\focus}[1]{{\cFocus#1}}
\newcommand{\cMath}{\usebeamercolor[fg]{math text}}
\newcommand{\cIdent}{\usebeamercolor[fg]{math text}}
\newcommand{\ident}[1]{{\cIdent#1}}
\newcommand{\cExp}{\cIdent}
\newcommand{\cBoring}{\color{grey}}
\newcommand{\boring}[1]{{\cBoring#1}}

%\newcommand{\ann}[1]{^{\cAnn #1}}
\newcommand{\unn}[1]{_{\cAnn #1}}
\newcommand{\annW}[1]{^{\hphantom{#1}}}
%\newcommand{\Ann}[1]{{\cAnn #1}}
\newcommand{\AnnW}[1]{\hphantom{#1}}
\newcommand{\ttAnn}[1]{\{{\cAnn #1}\}}
% ordinal annotation
\newcommand{\cOrd}{\cAnn}
\newcommand{\onn}[1]{^{\cOrd #1}}
\newcommand{\Onn}[1]{{\cOrd #1}}
\newcommand{\oforall}[1]{\forall\Onn{#1}.~}
\newcommand{\oexists}[1]{\exists\Onn{#1}.~}
\newcommand{\oapp}[1]{\,\Onn{#1}}
\renewcommand{\emph}[1]{{\cAnn#1}}
\newcommand{\OSize}{\Onn{Size}}
\newcommand{\oi}{\Onn{i}}
\newcommand{\odi}{\Onn{\$i}}
\newcommand{\oddi}{\Onn{\$\$i}}
\newcommand{\oj}{\Onn{j}}
\newcommand{\ohash}{\Onn{\#}}

% \newcommand{\kw}[1]{{\bf#1}}
\newcommand{\kwdata}{\kw{data}}
\newcommand{\kwcodata}{\kw{codata}}
\newcommand{\kwsized}{\kw{sized}}
\newcommand{\kwfun}{\kw{fun}}
\newcommand{\kwcofun}{\kw{cofun}}
\newcommand{\kwlet}{\kw{let}}
\newcommand{\kwfields}{\kw{fields}}
%\newcommand{\kw}{\kw{}}
%\newcommand{\kw}{\kw{}}
%\newcommand{\kw}{\kw{}}
\newcommand{\tinl}{\mathsf{inl}}
\newcommand{\tinr}{\mathsf{inr}}

\renewcommand{\rulename}[1]{#1}

% types
\newcommand{\cType}{\color{orange!60!black}}


% irrelevance
\newcommand{\irr}{\mathord{\bullet}}
\newcommand{\shirr}{\mathord{{\bullet}{\bullet}}}
\renewcommand{\erhyp}[3][\irr]{#1#2 \of #3}
\newcommand{\shirrhyp}{\erhyp[\shirr]}
\renewcommand{\erfunT}[4][\irr]{(\erhyp[#1]{#2}{#3}) \to #4}
\newcommand{\serfunT}[4][\irr]{(#1#2 : #3) \to #4}
\renewcommand{\resurrect}[1]{\irr^{-1}(#1)}

\begin{document}

\maketitle
%\begin{frame}
%  \titlepage
%\end{frame}

%\begin{frame}
%  \frametitle{Outline}
%  \tableofcontents
%  % You might wish to add the option [pausesections]
%\end{frame}

% What is the problem?  (Universe-levels)
% Set : Set inconsistent
% Introduce a stratification
% Set_i : Set_i+1
% higher-rank poly Data.Container.Combinator ∀

% What is the problem?  (Termination)
% first-class polymorphism for generic programming
%

\section{Introduction}


\begin{frame}%[fragile=singleslide]
  \frametitle{Story I: Stratification of Universes}
  \begin{itemize}
  \item $\Set : \Set$ (meaning $\Type : \Type$) inconsistent.
  \item Stratification $\Set_0 : \Set_1 : \Set_2 : \dots$
  \item Subtyping $\Set_0 \subset \Set_1 \subset \Set_2 \subset \dots$
  \item Polymorphism $f : \forall \ell \to (A : \Set_\ell) \to A \to A$.
  \item Level expressions are not unique: $f\,0\,\Nat\,x$ vs. $f\,1\,Nat\,x$.
  \item Get in the way of equality: $f\,0\,\Nat\,x = f\,1\,Nat\,x$?
  \end{itemize}
\end{frame}

\begin{frame}%[fragile=singleslide]
  \frametitle{Story II: Stratification of Datatypes}
  \begin{itemize}
  \item $\tfix : ((\Nat \to C) \to (\Nat \to C)) \to \Nat \to C$
    inconsistent
  \item Stratification $\Nat^0 \subset \Nat^1 \subset \Nat^2 \subset \dots \subset \Nat^\infty$ \\
    $\tfix : (\forall i \to (\Nat^i \to C) \to (\Nat^{i+1} \to C))
    \to \forall i \to \Nat^i \to C$
  \item Polymorphism $t : (\forall i \to \Tree^i \to \Tree^i) \to
    \forall i \to \Tree^i \to \Tree^\infty$
  \item Size expressions are not unique.
  \item Get in the way of equality proofs.
  \end{itemize}
\end{frame}

\begin{frame}[fragile=singleslide]
  \frametitle{Sizes in the Way of Agda}
\input{latex/MonusDiagError}
\aNat \\[2ex]

\aPredTy \\[2ex]

\aMonus
\end{frame}

\begin{frame}%[fragile=singleslide]
  \frametitle{What if the offender was absent?}
  \begin{itemize}
  \item Curry-style quantification:
\[
  \ru{t_1 = t_2 : \forall i \to A\,i
    }{t_1 = t_2 : A\,a}
\]
  \vspace{-1ex}
  \item Great, but want Church-style syntax for type-checking.
\[
  \ru{t_1 = t_2 : \forall i \to A\,i
    }{t_1 \ann {a_1} = t_2 \ann {a_2} : A\,a}
\]
  \vspace{-1ex}
  \item Semantically, instantiations $a_1$ and $a_2$ are \emph{irrelevant}.
  \item ``Semantic'' typing rule (\emph{not} \textit{type checking} rule):
\[
  \ru{t : \forall i \to A\,i
    }{t \ann a : A\,b}
\]

  \end{itemize}
\end{frame}

\begin{frame}%[fragile=singleslide]
  \frametitle{(When) does this make sense?}
\vspace{-2ex}
  \begin{itemize}
  \item ICC($^*$)  (Miquel, Barras, Bernardo, Sheard, Mishram-Linger): \\
    ``Always, under erasure''
%  \item Not compatible with type-directed $\eta$ with unit type $\star : \ONE$:
  \item What about $\eta$ for unit type $\star : \ONE$ with $t = t' : \ONE$?
\[
\ru{f : \forall X \to (X \to X) \to C
  }{
    f \ann {A \to A} (\lambda x.\,x) =
    f \ann \ONE (\lambda x.\,\star) =
    f \ann {A \to A} (\lambda x.\, \lambda y.\,x\,(x\,y))
    % \\
    % f \ann {A \to A} (\lambda x \lambda y.\,x\,y) =
    % f \ann A (\lambda x.\,x) =
    % f \ann \ONE (\lambda x.\,\star)
  }
\]
\vspace{-1ex}
  \item Our restriction: the choice of instantiation may not affect
    $\eta$.
  \item $\eta$ is directed by the type \emph{shape}: function type,
    unit type, $\Sigma$-type, other type.
  \item Here: $X$ is \emph{not} irrelevant for the shape of $(X \to X) \to C$.
  \end{itemize}
\end{frame}

\begin{frame}%[fragile=singleslide]
  % \frametitle{(Shape-)irrelevance}
  % \begin{itemize}
  % \item Formation of irrelevant function types.
  \frametitle{Dependent irrelevant function types}
    \begin{enumerate}
    \item The cautious (Pfenning, LiCS 2001)
\[
  \ru{\erhyp x A \der B : \Set
    }{\der \erfunT x A B : \Set}
\]
    \item The daredevil (Miquel, LiCS 2000)
\[
  \ru{x \of A \der B : \Set
    }{\der \erfunT x A B : \Set}
\]
    \item ``lagom'' (Abel, AIM 2011)
\[
  \ru{\shirrhyp x A \der B : \Set
    }{\der \erfunT x A B : \Set}
\]
    \end{enumerate}
%  \end{itemize}
\end{frame}

\begin{frame}%[fragile=singleslide]
  \frametitle{Shape-irrelevance}
\vspace{-2ex}
  \begin{itemize}
  \item These are shape-irrelevant in $i$
\[
\begin{array}{@{\shirr i \der~}l}
  \Nat^i
\\
  \Nat^i \to \Nat^{i+1}
\\
  \tifthenelse b {\Set_i} {\Set_i \to \Set_i}
\\
  (x : \Nat^i) \to \mathsf{Vec}\,A\,x
\end{array}
\]
  \item These are \emph{not} shape-irrelevant:
\[
\begin{array}{@{\shirr}l@{~\not\der~}l}
  b \of \Bool  & \tifthenelse b {\Set_0} {\Set_0 \to \Set_0}
\\
  X \of \Set_0 & X
\\
  X \of \Set_0 & X \to X
\\
\end{array}
\]
  \end{itemize}
\end{frame}

\begin{frame}%[fragile=singleslide]
  \frametitle{Defining Shapes (in the Model)}
\vspace{-3ex}
  \begin{itemize}
  \item Base types of the same shape:
\[
  \ONE \approx \ONE
\qquad
  \Nat^i \approx \Nat^j
\qquad
  \Set_\ell \approx \Set_{\ell'}
\]
\vspace{-2ex}
  \item Function types:
\[
  \ru{A_1 \approx A_2 \qquad B_1(a) \approx B_2(a) \mforall a \in \alert{A_1}
    }{\funT x {A_1}{B_1(x)} \approx \funT x {A_2}{B_2(x)}}
\]
\vspace{-2ex}
  \item \emph{Not symmetric!}
\[
  \mathit{template} \shape \mathit{shape}
\]
\vspace{-2ex}
  \item No syntactic judgement for \textit{same shape}. :(
  \end{itemize}
\end{frame}


\begin{frame}%[fragile=singleslide]
  \frametitle{Finally, Normalization by Evaluation (NbE)!}
\vspace{-2ex}
  \begin{itemize}
  \item TA-NbE (TA = Type Assignment $\not=$ Thorsten Altenkirch)
  \item Values $a$ are (extended) weak head normal forms.
  \item Relations $a \in A$ and $a = a' \in A$ between whnfs.
  \item Reflecting neutral term $u$ as value $\up A u \in A$:
\[
\begin{array}{rll}
   (\up {\sfunT x A B(x)} u)(a) & = & \up {B(a)} (u\,\down A a)
\\
   (\up {\serfunT x A B(x)} u)(a) & = & \up {B(a)} (u \ann{\down A a})
\end{array}
\]
\vspace{-2ex}
  \item Reifying value $a \in A$ as normal term $\down A a$:
\[
\begin{array}{lll}
  \down \ONE a & = & \star \\
  \down {\sfunT x A B(x)} f & = & \lambda y.\, \down{B(\up A y)} f(\up A y)
\end{array}
\]
  \end{itemize}
\end{frame}


\begin{frame}%[fragile=singleslide]
  \frametitle{Reflection and Reification}
  \begin{theorem}
    Let $A \shape A_1$ and $A \shape A_2$.
    \begin{enumerate}
    \item If $u_1$ and $u_2$ are equal neutrals then $\up {A_1} u_1 = \up {A_2} u_2 \in A$.
    \item If $a_1 = a_2 \in A$ then $\down {A_1} a_1$ and $\down {A_2} a_2$ are equal normal forms.
    \end{enumerate}
  \end{theorem}
  \begin{proof}
    \begin{itemize}
    \item
    Goal $\up{\color{darkred}\serfunT x {A_1}{B_1(x)}} u_1 = \up{\color{darkred}\serfunT x {A_2}{B_2(x)}} u_2 \in \color{darkred}\erfunT x A {B(x)}$.
    \item
    Assume $a_1 \in A$ and $a_2 \in A$.
    \item
    Show $(\up{\serfunT x {A_1}{B_1(x)}} u_1)(a_1) = (\up{\serfunT x {A_2}{B_2(x)}} u_2)(a_2) \in B(a_1)$.
    \item
    Show $\up{\color{darkred}B_1(a_1)} (u_1 \ann {\down{A_1} a_1}) = \up{\color{darkred}B_2(a_2)} (u_2 \ann {\down{A_2}a_2}) \in \color{darkred}B(a_1)$.
    \end{itemize}
    Since $B_1(x)$ and $B_2(x)$ are shape-irrelevant in $x$, we apply the induction hypothesis with
    $B(a_1) \shape B_1(a_1)$ and $B(a_1) \shape B_2(a_2)$.
  \end{proof}
\end{frame}


\begin{frame}%[fragile=singleslide]
  \frametitle{Decidability}
  \begin{itemize}
  \item NbE decides definitional equality.
  \item Type checking (bidirectional) decidable with rule:
\[
  \ru{\Gamma \der t \jinf \erfunT x U T(x) \qquad
      \resurrect \Gamma \der u : U
    }{\Gamma \der t \ann u \jinf T(u)}
\]
  \end{itemize}
\end{frame}

\begin{frame}[fragile=singleslide]
  \frametitle{Sizes out of the Way!}
\vspace{-3ex}
\input{latex/MonusDiag}
\aOpt \\[1ex]
\aNat \\[2ex]

\aPredTy \\[2ex]

\aMonus
\end{frame}

\begin{frame}%[fragile=singleslide]
  \frametitle{Conclusions}
  \begin{itemize}
  \item Irrelevance modality allows us to ignore sizes where they just help the type checker.
\[
  \suc {{\color{grey}\mathit{ignoreMe}}} n : \Nat^{\alert{\mathit{dontIgnoreMe}}}
\]
\vspace{-2ex}
  \item Codomain of dependent irrelevant function type needs to be \emph{shape-irrelevant}.
  \item Better semantics for these modalities? (Vezzosi, Nuyts)
  \item Beware type-theorist! More modalities are coming your way!
  \end{itemize}
\end{frame}


\begin{frame}%[fragile=singleslide]
\vfill
\begin{quotation}
  These modalities are horribly complicated, can't we get rid of them?
\end{quotation}
\begin{flushright}
---Phil Wadler (Leuven, 2017-05-18)
\end{flushright}
\vfill
\end{frame}


%%%%%%%%%%%%%%%%%%%%%%%%%%%%%%%%%%%%%%%%%%%%%%%%%%%%%%%%%%%%%%%%%%%%%%
%%%%%%%%%%%%%%%%%%%%%%%%%%%%%% END DOC %%%%%%%%%%%%%%%%%%%%%%%%%%%%%%%
%%%%%%%%%%%%%%%%%%%%%%%%%%%%%%%%%%%%%%%%%%%%%%%%%%%%%%%%%%%%%%%%%%%%%%

% \bibliography{short}

\end{document}



\begin{frame}%[fragile=singleslide]
  \frametitle{}
  \begin{itemize}
  \item
  \end{itemize}
\end{frame}


\begin{frame}%[fragile=singleslide]
  \frametitle{}
  \begin{itemize}
  \item
  \end{itemize}
\end{frame}


\begin{frame}%[fragile=singleslide]
  \frametitle{}
  \begin{itemize}
  \item
  \end{itemize}
\end{frame}


\begin{frame}%[fragile=singleslide]
  \frametitle{}
  \begin{itemize}
  \item
  \end{itemize}
\end{frame}


\begin{frame}%[fragile=singleslide]
  \frametitle{}
  \begin{itemize}
  \item
  \end{itemize}
\end{frame}


\begin{frame}%[fragile=singleslide]
  \frametitle{}
  \begin{itemize}
  \item
  \end{itemize}
\end{frame}


\begin{frame}%[fragile=singleslide]
  \frametitle{}
  \begin{itemize}
  \item
  \end{itemize}
\end{frame}





%%% Local Variables:
%%% mode: latex
%%% TeX-master: t
%%% End:
